%!TEX root = ../bachelorthesis.tex


\chapter{Einleitung}
\markboth{Einleitung}{}
\label{sec:Einleitung}

\blindtext


\chapter{Digitalisierung in der Verwaltung}

Kurze Beschreibung von technischer Basis von KitaNet. Was sind soziale Netzwerke?
Hier sollen die verschiedenen Einsatzszenarien von KitaNet abgebildet werden

\section{Kommunikationsplattform}
Kann Kommunitkation durch KitaNet verbessert werden? Verstehen alle Mitarbeiter die Plattform?
Sind Schwellenängste vorhanden?

\subsection{Kommunikation mit Eltern}
Sollten Eltern Zugriff bekommen? Was sind Vorteile eines solchen Systems? Exkludiert man dadurch Eltern, die sich kein Smartphone leisten können, oder es nicht bedienen können?

\section{Dokumentenmanagement}

Was sind die Anforderungen an ein Dokumentenmanagement? 
Wozu bedarf es eines "Gelesen"-Status? 

\chapter{Digitalisierung in der Arbeit am Kind}

Was sind die Aufgaben eines Erziehers? Welche Vorgaben gibt es bzgl. Pädagogik im Bereich Medien?
Was ist überhaupt Medienpädagogik, bzw. was ist die gelebte Realität?

\section{Kleine Kinder mit dem Tablet}

Welche Bedenken gibt es Kinder mit Technik in Berührung zu bringen? Welche Argumente sprechen dafür, welche 
dagegen?

\section{Softwarebasierte Falldokumentation}

Erläuterung von Entwicklungstagebüchern im Rahmen der pädagogischen Kinderbetreuung 
Aufzeigen von Risken, dass bei zu großer Formalisierung der Dokumentation Teile der Realität ausgeblendet werden, die der Programmierer nicht bedacht hat \citep[vgl.][38 ff]{weber2017}. 

\section{Programmieren des kindlichen Roboters}

Darstellung und Analyse eines Projektes der Stftung "Haus der kleinen Forscher" zum Thema Formalisierung und Programmierung\citep[vgl.][]{roboter}.

\chapter{Die Rolle der Sozialinformatik}

Argumentation über Kreidenweis hin zu einem praktischen Ansatz des Ansprechpartners im Betreuungsdreieck zwischen Kind, Eltern und Einrichtung.

\section{Die Informatik aus der Sicht der Pädagogik}

Literaturbeispiele von Einsatzszenarien, in denen Päsagogen einen Informatiker einsetzen würden. \textquote{Bezugnehmend auf die eingangs erwähnte Anekdote wirkt es vor diesem HIntergrund gar nicht so befremdlich, dass sich Studierende der Bildungswissenschaften bei der Erstellung einer PowerPoint-Präsentation die Hilfe und Unterstützung eines Informatikers wünschen}\citep[][41]{Klar2015}. 

\chapter{Fazit}

\blindtext
