%!TEX root = ../bachelorthesis.tex


\chapter{Einleitung}
\markboth{Einleitung}{}
\label{sec:Einleitung}

\blindtext


\chapter{KitaNet}

Kurze Beschreibung von technischer Basis von KitaNet. Was sind soziale Netzwerke?
Ablauf beim Grundstellen von Passwörtern wird erläutert.

\section{Hardware}
Beschreibt den technischen Aufbau von KitaNet in der Kita selbst.
Abweichungen vom Aufbau für Studienprojekt werden erläutert. Was weicht ab?
Verhindert die für dieses Bachelorarbeit eingesetzte Hardware eine Umsetzung in der Kita?

\section{HumHub}
Kurze Darstellung der KitaNet-Software. Überblick über Funktionen von HumHub (Module, Spaces, etc.)

\section{LDAP}
Begriffserklärung und Funktionsbeschreibung Nutzerverzeichnis. 
Wie arbeitet das \ac{ldap} mit KitaNet zusammen?

\chapter{Mail-Server}

\section{Anforderungen/Nutzungsszenarien}
\subsection{Nutzungsszenarien}
Welche Szenarien soll der \ac{mta} abdecken? Welche Funktionen sind notwendig (\zb nur interne Mails, keine Erreichbarkeit von außen)? Was ist bei der Lizensierung zu beachten? Gibt es weitere wichtige Aspekte? Der SMTP-Server soll mit dem eingesetzten IMAP-Client Dovecot zusammenarbeiten.
\subsection{Testfälle}
Aufgrund der beschriebenen Nutzungsszenarien werden Testfälle formuliert, die den Erfolg des Projektes kennzeichnen. 

\section{SMTP-Software}
Vergleich von SMTP-Server-Software für den Einsatz auf Ubuntu. 
\subsection{postfix}
Erfüllt die Software die gestellten Anforderungen? Was spricht gegen einen Einsatz?
\subsection{EmailSuccess}
Inhaltlich wie oben.

\subsection{Entscheidung}
Welche SMTP-Software wurde gewählt? 

\chapter{Installation und Tests}

\section{Einrichtung und Anbindung SMTP an LDAP}
Wie steuert das \ac{ldap} den SMTP-Server? Wie funktioniert der Informationsaustausch (neue Nutzer, etc)?

\section{Tests}
Allgmeines zu den durchgeführten Tests. Kam es zu Problemen bei der Testung?
\subsection{Dokumenation der einzelnen Tests}
Wurde der Test bestanden? Musste der Test unerwartet an die Gegebenheiten angepasst werden?

\chapter{Fazit}

\blindtext
