%!TEX root = ../bachelorthesis.tex


\chapter{Einleitung}
\markboth{Einleitung}{}
\label{sec:Einleitung}

\blindtext


\chapter{KitaNet}

Kurze Beschreibung von technischer Basis von KitaNet. Was sind soziale Netzwerke?
Ablauf beim Grundstellen von Passwörtern wird erläutert.

\section{Hardware}
Beschreibt den technischen Aufbau von KitaNet in der Kita selbst.
Abweichungen vom Aufbau für Studienprojekt werden erläutert. Was weicht ab?
Verhindert die für dieses Bachelorarbeit eingesetzte Hardware eine Umsetzung in der Kita?

\section{HumHub}
Kurze Darstellung der KitaNet-Software. Überblick über Funktionen von HumHub (Module, Spaces, etc.)

\section{LDAP}
Begriffserklärung und Funktionsbeschreibung Nutzerverzeichnis. 
Wie arbeitet LDAP mit KitaNet zusammen?

\chapter{SMTP-Server}

\section{Anforderungen}
Welche Szenarien soll der Mailserver abdecken? Welche Funktionen sind notwendig? 

\section{Wie funktioniert E-Mail?}
Beschreibung der Grundfunktionalität


\section{SMTP-Server}
Vergleich von SMTP-Server-Software für den Einsatz auf Ubuntu

\section{Einrichtung und Anbindung SMTP an LDAP}
Wie "steuert" das LDAP den SMTP-Server? Wie funktioniert der Informationsaustausch (neue Nutzer, etc)?

\chapter{E-Mail aus Nutzersicht}

\section{IMAP-Client}
Vergleich von Dovecot und Roundcube auf Kompatibiliät mit System

\section{Einrichtung IMAP}
"Gewinner" des Vergleichs wird implementiert.
Anknüpfung an LDAP (ggf. eigenes Unterkapitel?)


\chapter{Fazit}

\blindtext
