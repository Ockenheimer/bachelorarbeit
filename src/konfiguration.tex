\usepackage{amsmath}
\usepackage{amsfonts}
\usepackage{amssymb}
\usepackage{xspace}
\usepackage[normalem]{ulem}
\usepackage{graphicx}
\usepackage{color, colortbl}
\usepackage[table]{xcolor}
\usepackage{nicefrac}
\usepackage{float}


%https://tex.stackexchange.com/questions/97474/how-to-add-lstlistoflistings-to-the-table-of-contents
\usepackage[nottoc]{tocbibind}

%%%%%%%%%%%%%%%%%%%%%%%%%%%%%%%%%%%%%%%%%%%%%%%%%%%%%%%%%%%%%%%%%%%%%%%%%%%%%%%%%%%%%%%%%%
% In diesem Abschnitt gibst du deine persönlichen Daten sowie den Kontext deiner Arbeit an
% das \xspace am Ende der Einträge muss vorhanden bleiben. Es regelt das Whitespace-
% handling
%%%%%%%%%%%%%%%%%%%%%%%%%%%%%%%%%%%%%%%%%%%%%%%%%%%%%%%%%%%%%%%%%%%%%%%%%%%%%%%%%%%%%%%%%%
% persönliche Daten
\newcommand{\vorname}{Markus\xspace}
\newcommand{\nachname}{Schäfer\xspace}
\newcommand{\emailadresse}{markus.schaefer@et.hs-fulda.de\xspace}
\newcommand{\matrikelnummer}{945228\xspace}


% Arbeitsspezifische Daten
\newcommand{\titel}{mail@kitanet\xspace}
\newcommand{\untertitel}{Implementierung eines internen E-Mail-Dienstes als Funktionserweiterung eines sozialen Netzwerkes\xspace}
\newcommand{\modulname}{Bachelor-Thesis\xspace}
\newcommand{\pruefer}{Prof. Dr. Uwe Werner\xspace}
\newcommand{\zweitpruefer}{Dipl. Ing. Rouven Braden\xspace}
\newcommand{\semester}{WiSe 2021\xspace}
\newcommand{\ortderarbeit}{Ockenheim\xspace}
%%%%%%%%%%%%%%%%%%%%%%%%%%%%%%%%%%%%%%%%%%%%%%%%%%%%%%%%%%%%%%%%%%%%%%%%%%%%%%%%%%%%%%%%%%

%Um einzelne Seite im Querformat zu verwenden
\usepackage{pdflscape}

% Damit Tabellen über die ganze Seitenbreite gehen
% https://texwelt.de/wissen/fragen/446/wie-bekomme-ich-eine-tabelle-in-textbreite
\usepackage{booktabs}
\usepackage{tabularx}
% Abstand zwischen Tabelle und Überschrift
% http://www.chemieonline.de/forum/showthread.php?t=101576
%\setlength{\belowcaptionskip}{0.3em}
\usepackage[center]{caption}
\newcommand*{\thead}[1]{\multicolumn{1}{c}{\bfseries #1}}
       
\usepackage{tikz}
\usepackage{pgfplots, pgfplotstable}

%Hurenkinder uns Schusterjungen
\clubpenalty10000
\widowpenalty10000
\displaywidowpenalty=10000

\usepackage{hyperxmp} % XMP-Daten fuer die PDF-Datei
%Gimmmick: Linked Kapitel und Inhaltsverzeichnis, sowie Referenzen
\usepackage[pdftex, pdfa]{hyperref}
\hypersetup{
    colorlinks,
    citecolor=black,
    filecolor=black,
    linkcolor=black,
    urlcolor=black,    
    pdftitle = {\titel},
    pdfauthor = {\vorname \nachname},
    pdfsubject = {\untertitel},
    pdfkeywords = {\titel \untertitel \matrikelnummer},
    pdflang = de,
    %bookmarks = true,
    pdfdisplaydoctitle = true,
    colorlinks = true,
    plainpages = false,
    %allcolors = black,
    hypertexnames = false,
    %pdfpagelabels = true,
    %hyperindex = true,
    unicode = true,
    pdfcaptionwriter = {\textsl{\vorname \nachname}},
    pdfcontactaddress = {In den Weider 4},
    pdfcontactcity = {Ockenheim},
    pdfcontactpostcode = {55437},
    pdfcontactcountry = {Deutschland},
    pdfcontactregion = {RLP},
    pdfcontactemail = {\emailadresse},
    pdfcontactphone = {0176/20479746},
    pdfcontacturl = {http://www.hs-fulda.de},
    pdfmetalang = {de},
}

\usepackage{prettyref}
\usepackage{titleref}

%Schriftart

%\renewcommand*\rmdefault{cmdh}
%\usepackage{lmodern}
%\usepackage{mathptmx}
%\usepackage[scaled=.90]{helvet}
%\usepackage{courier}
%\usepackage{bera}
%\usepackage{boisik}

%Inhaltsverzeichnis
\usepackage[tocgraduated]{tocstyle}
\usetocstyle{allwithdot}
\setcounter{tocdepth}{3}

% Formatierungshilfen: http://wwws.htwk-leipzig.de/~myagovki/latex/formatierungshilfen/

\usepackage[ddmmyyyy]{datetime}
\renewcommand{\dateseparator}{.}


% Für Kopf und Fußzeile: https://esc-now.de/_/latex-individuelle-kopf--und-fusszeilen/?lang=en
\usepackage[
  headsepline, plainheadsepline,
  footsepline, plainfootsepline
]{scrlayer-scrpage}
\pagestyle{scrheadings}
\clearscrheadfoot

    
\ihead*{
   \itshape \leftmark
}
\ohead*{\vorname \nachname}
\ifoot*{\today}
\ofoot*{\pagemark}





% anderthalbfacher Zeilenabstand
\usepackage[onehalfspacing]{setspace}

%Geometry der Seite festlegen
\usepackage{geometry}
\geometry{
  left=2.5cm,
  right=2.5cm,
  top=2.5cm,
  bottom=2.5cm,
  headheight=33pt
}

%Für Gestaltung des Layouts
\usepackage{blindtext}

%Bilbliothek und Literatur einbinden
\usepackage[style=authoryear-ibid,natbib=true,bibencoding=utf8,backend=biber,sorting=nty]{biblatex}
\usepackage[babel,german=guillemets]{csquotes}
\defbibheading{head}{{\chapter*{Literaturverzeichnis}\addcontentsline{toc}{chapter}{Literaturverzeichnis}}}
\addbibresource{bib/literatur.bib}

%https://texwelt.de/fragen/19153/pdf-dateien-in-anhang-einbinden-aber-formatierung-des-vorhergehenden-dokuments-ubernehmen
\usepackage{pdfpages}

\newcommand{\zb}{z.\,B.\xspace}
\newcommand{\ua}{u.\,a.\xspace}
\newcommand{\oge}{o.\,g.\xspace}
\newcommand{\dah}{d.\,h.\xspace}
\newcommand{\bzw}{bzw.\xspace}
\newcommand{\bzgl}{bzgl.\xspace}
\newcommand{\ggf}{ggf.\xspace}
\newcommand{\bspw}{beispielsweise\xspace}

%https://stackoverflow.com/questions/3175105/writing-code-in-latex-document
\newcommand{\code}[1]{\texttt{#1}}

%https://tex.stackexchange.com/questions/115467/listings-highlight-java-annotations
\usepackage{listings}
\usepackage{inconsolata}

\usepackage{color}

\definecolor{pblue}{rgb}{0.13,0.13,1}
\definecolor{pgreen}{rgb}{0,0.5,0}
\definecolor{pgrey}{rgb}{0.46,0.45,0.48}
\definecolor{mauve}{rgb}{0.58,0,0.82}

\usepackage{listings}

\lstset{
  frame=t,
  language=Java,
  aboveskip=1mm,
  belowskip=1mm,
  columns=flexible,
  showspaces=false,
  showtabs=false,
  numbers=left,
  breaklines=true,
  showstringspaces=false,
  breakatwhitespace=true,
  commentstyle=\color{pgreen},
  keywordstyle=\color{pblue},
  stringstyle=\color{mauve},
  basicstyle={\small\ttfamily},
  breaklines=true,
  breakatwhitespace=true,
  tabsize=2,
  moredelim=[il][\textcolor{pgrey}]{$$}, %$$ 
  moredelim=[is][\textcolor{pgrey}]{\%\%}{\%\%}
}

\renewcommand{\lstlistoflistings}{\begingroup
\tocfile{\lstlistingname}{lol}
\endgroup}

% Silbentrennung
\hyphenation{Soft-ware-pro-jekt}
\hyphenation{Soft-ware}
