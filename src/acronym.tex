%!TEX root = ../bachelorthesis.tex

\chapter*{Abkürzungsverzeichnis}
\markboth{Abkürzungsverzeichnis}{}
%\addcontentsline{toc}{chapter}{Abkürzungsverzeichnis}

\begin{acronym}
%a
\acro{arpanet}[ARPANET]{\textit{Advanced Research Projects Agency Network}}
%b
%c
\acro{cn}[cn]{\textit{Common Name}}
%d
\acro{dc}[dc]{\textit{Domain Component}}
\acro{dit}[DIT]{\textit{Directory Information Tree}}
\acro{dn}[dn]{\textit{Distinguished Name}}
%e
\acro{epl}[EPL]{\textit{Eclipse Public License - v 2.0}}
\acro{esmtp}[ESMTP]{\textit{SMTP Service Extensions}}
%f
\acro{fqdn}[FQDN]{\textit{fully qualified domain name}}
%g
\acro{gpl}[GPL]{\textit{GNU General Public Licence}}
%h
%i
\acro{iana}[IANA]{\textit{Internet Assigned Numbers Authority}}
\acro{ieee}[IEEE]{\textit{Institute of Electrical and Electronics Engineers}}
\acro{ietf}[IETF]{\textit{Internet Engineering Task Force}}
\acro{imap}[IMAP]{\textit{Internet Message Access Protocoll}}
\acro{ipl}[IPL]{\textit{IBM Public License (IPL 1.0)}}
%j
%k
\acro{kita}[Kita]{\textit{Kindertagesstätte}}
%l
\acro{lan}[LAN]{\textit{Local-Area-Network}}
\acro{ldap}[LDAP]{\textit{Lightweight Directory Access Protocol}}
wenn nicht explizit abweichend im Text erläutert, steht der Begriff als Synonym für die Implementation von LDAP-Protokoll und LDAP-Datenbank zum Zweck der Rechteverwaltung
\acro{lmtp}[LMTP]{\textit{Local Mail Transfer Protocol}}
%m
\acro{mda}[MDA]{\textit{Mail Delivery Agent}}
\acro{mime}[MIME]{\textit{Multipurpose Internet Mail Extensions}}
\acro{mta}[MTA]{\textit{Mail Transport Agent}}
%n
\acro{nas}[NAS]{\textit{Netwok Attached Storage}}
%o
\acro{ou}[OU]{\textit{Organisational Unit}}
%p
\acro{php}[PHP]{\textit{PHP: Hypertext Preprocessor}}
%q
%r
\acro{re}[RE]{\textit{Requirements Engineering}}
\acro{rfc}[RFC]{\textit{Request for Comments}}
%s
\acro{sasl}[SASL]{\textit{Simple Authentication and Security Layer}}
\acro{smtp}[SMTP]{\textit{Simple Mail Transfer Protocol}}
\acro{srs}[SRS]{\textit{Software Requirements Specifications}} 
(deutsch: Anforderungsdokument)
%t
\acro{tcp}[TCP/IP]{\textit{Transmission Control Protocol/Internet Protocol}}
%U
%v
\acro{vm}[VM]{\textit{virtuelle Maschine}}
%w
%x
%y
%z
\end{acronym}