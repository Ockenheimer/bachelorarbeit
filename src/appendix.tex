%!TEX root = ../bachelorthesis.tex
\appendix
\renewcommand\chaptername{Anhang}


\chapter{Anforderungsdokument}
\markboth{A. Ein Anhang}{}
\label{ch:Anhang}

\section{Einführung}
\subsection{Zweck des Dokuments}
Dieses Dokument definiert die Anforderungen an das Projekt \"Einführung eines SMTP-Servers". 
Dieses Dokument richtet sich an Entwickler die mit diesem Projekt betraut sind.

\subsection{Umfang}
Die zu installierende Software soll einen SMTP-Server bereitstellen, um E-Mail-Kommunikation innerhalb des Netzwerks der Einrichtung zu ermöglichen (vgl. RFC 821). Der Mailverkehr wird zunächst nur innerhalb des eigenen Netzwerks gewährleistet. Eine Öffnung nach außen ist zukünftig angestrebt, aber zunächst nicht Teil des Projekts.

\subsection{Definitionen, Akronyme, Abkürzungen}
\label{ch:Definition}
Eine Liste von Abkürzungen, die in diesem Dokument verwendet werden.
\begin{itemize}
	\item LDAP - Lightweight Directory Access Protocol
	\item LTS - Long Time Support
	\item RFC - Request for Comments
	\item SMTP - Simple Mail Transfer Protocol
\end{itemize}

Die Anforderungen werden in drei Stufen der Notwendigkeit unterschieden und entsprechend priorisiert. Die Zugehörigkeit der Anforderung zu den Stufen werden mit der Formulierung der Anforderung angezeigt.\\
Anforderungen beginnend mit:
\begin{itemize}
	\item \textit{Das System muss} oder \textit{Das System darf keine} gehören zur höchsten Stufe.
	\item \textit{Das System soll} oder \textit{Das System darf} gehören zur mittleren Stufe.
	\item \textit{Das System kann} gehören zur niedrigen Stufe.
\end{itemize}

\subsection{Referenzen}
Referenzierte Dokumente, die in diesem Dokument verwendet werden.
\begin{itemize}
	\citereset
	\item RFC 821 \\
	https://tools.ietf.org/html/rfc821 
	\item RFC 2821\\
	https://tools.ietf.org/html/rfc2821
	\citereset
\end{itemize}



\subsection{Übersicht}
Dieses Anforderungsdokument enthält die Anforderungen und Spezifikationen, die an den SMTP-Server gestellt werden.
Sie dienen als Entscheidungsgrundlage für die anzuschaffende Software.

\section{Produktübersicht}
\subsection{Produktperspektive}
Das finale System bildet eine virtuelle Maschine (im weiteren \textit{der Server}) innerhalb eines QNAP-NAS mit 4GB Arbeitsspeicher.
Auf dem Server wird als Betriebssystem Ubuntu 20.04 LTS eingesetzt. Auf diesem System wird ebenfalls folgende relevante Software eingesetzt:
\begin{itemize}
	\item Open-LDAP ver. 2.4.49
	\item Apache2 ver. 2.4.41
	\item Dovecot ver. 2.3.7.2
	\item PHP ver. 7.4
\end{itemize} 
Der Server ist für Laptops innerhalb des Netzwerks erreichbar. Ein direkter Zugriff zur Hardware ist grds. nicht vorhanden.

Die Software ist erreichbar über Port 25 des Servers.  

\subsection{Produktfunktionen}
Die Software verwaltet die Entgegennahme und Auslieferung von Datenpaketen mit dem IMAP-Client Dovecot.
Zur Ermittlung bekannter Mail-Adressen gleicht die Software ihren Bestand mit dem lokalen \ac{ldap} ab.
Die Software besitzt den vollen Funktionsumfang von \ac{smtp} (vgl. RFC 2821).

\subsection{Nutzereigenschaften}
Der Nutzer hat keine direkte Interaktion mit der Software. Kontakt besteht ausschließlich mittelbar und automatisiert über den IMAP-Client Dovecot
\subsection{Einschränkungen}
Die Software teilt sich den vorhandenen Arbeitsspeicher mit einer per PHP realisierten Webseite und einem LDAP-Server. Somit darf die durchschnittliche RAM-Belegung nicht größer als 512 MB sein.
\subsection{Annahmen und Abhängigkeiten}
Das System bleibt zumindest bis April 2022 auf dem Betriebssystem Ubuntu 20.04 LTS. Sollte ein Plattformwechsel angestrebt werden, muss vorher die Funktionalität getestet werden.
\subsection{verzögerte Anforderungen}
keine 
\section{Spezifische Funktionen}

\subsection{externe Schnittstellen}
\subsubsection*{Benutzerschnittstellen}
Nicht vorgesehen.
\subsubsection*{Hardwareschnittstellen}
Nicht vorgesehen.
\subsubsection*{Softwareschnittstellen}
Nicht vorgesehen.
\subsubsection*{Kommunikationsschnittstellen}
Erreichbarkeit über Port 25 zum Austausch von \ac{lmtp}-Befehlen mit IMAP-Client auf dem selben System. Die Öffnung des Ports nach außen soll für zukünftigen Austausch von \ac{smtp}-Befehlen mit entfernten \ac{smtp}-Servern vorgesehen werden.
\subsection{Funktionen}
\subsubsection*{(E)SMTP-Kommunikation}
\textbf{Zweck der Funktion} \\
Das System muss (E)\ac{smtp}- und \ac{lmtp}-Befehle entgegennehmen und verarbeiten. \\
\textbf{Auslösung/Reaktion der Funktion} \\
Auslösung: \\
Kontaktaufnahme durch Zugriff anderer Programme auf Port 25\\
Reaktion:\\
Einleitung des (E)SMTP-Protokolls MAIL gem. S.4 RFC 821 \\
\textbf{Mit Funktion verbundene Anforderungen} \\
Versenden und Empfangen von E-Mails werden ermöglicht.\\
\subsubsection*{Abgleich LDAP}
\textbf{Zweck der Funktion} \\
Das System soll einen aktuellen Datenbestand über im LDAP angelegte Empfänger besitzen.\\
\textbf{Auslösung/Reaktion der Funktion} \\
Auslösung:\\
Automatisierter Anstoß zur Aktualisierung der Daten.\\
Reaktion:\\
Aufnahme neuer Empfänger, Löschung von nicht existenten Adressen.\\
\textbf{Mit Funktion verbundene Anforderungen}\\
Es werden nur Mails von Empfängern verarbeitet, die in der Datenbank des LDAP erfasst sind.\\
\subsection{Leistungsanforderung}
Das System soll Mails so schnell wie möglich, jedoch in maximal 15 Sekunden beim Empfänger abliefern. Eine Echtzeit-Verarbeitung ist nicht notwendig.\\
Das System soll 100\% der Mails beim jeweiligen Empfänger abliefern.\\
Das System soll Mails mit Dateianhängen bis zu 25 MB verarbeiten können.\\
\subsection{Design-Einschränkungen}
Das System muss mit dem \ac{imap}-Client Dovecot zusammenarbeiten.
\subsection{Softwaresystem-Eigenschaften}
\label{sec:zuverl}
\textbf{Zuverlässigkeit} \\
Das System muss nach einem Neustart des Servers ohne Nutzerzutun automatisch starten.\\
\textbf{Verfügbarkeit} \\
Das System soll jederzeit zur Verfügung stehen.\\
\textbf{Sicherheit} \\
Das System soll einen unberechtigten Zugriff Dritter auf die verarbeiteten E-Mails verhindern.\\
\textbf{Wartbarkeit} \\
Das System soll, ohne Veränderung der Systemumgebung, wartungsfrei sein.\\
Das System soll über eine detaillierte Dokumentation verfügen.\\
\textbf{Portierbarkeit} \\
Das System soll grundsätzlich auf andere Plattformen portierbar sein.\\

\subsection{andere Anforderungen}
\subsubsection*{Anschaffungskosten}
Das System darf keine Anschaffungskosten auslösen.
\subsubsection*{Betriebskosten}
Das System darf keine laufenden Betriebskosten durch Lizenzierung o.ä. auslösen.


\chapter{Testfälle}
\label{ch:Testfaelle}

\section{Testreihe 1 Versenden einer Mail}
\subsection{Testfall 1.1}
\begin{itemize}
	\item Testgegenstand:\\
	Versenden einer E-Mail an einen existierenden Empfänger
	\item Testkonfiguration:\\
	System ohne Last\\
	Mail-Sender \verb+mail@kitanet.local+ in System angelegt\\
	Mail-Empfänger \verb+testnutzer@kitanet.local+ in System angelegt
	\item Testbeschreibung:\\
	Versenden einer E-Mail über Roundcube mit Nutzer \verb+mail@kitanet.local+ \\ an \verb+testnutzer@kitanet.local+
	\item Bezug:\\
	Anforderung \textit{(E)SMTP-Kommunikation}
	\item Priorität:\\
	unbedingt erforderlich
	\item Details:\\
	Stellt sicher, dass Mails über den Server versendet werden können (Grundfunktionalität System).
	\item Soll-Ergebnis:\\
	E-Mail geht in Postfach des Empfängers ein.
	\item Ist-Ergebnis:\\
	E-Mail ging in Postfach des Empfängers ein.
	\item Bestanden:\\
	Ja
	\item Aus welcher Phase stammt der Fehler:\\
	Kein Fehler
	\item Kommentar:\\
	Keine Wartezeit spürbar.
	\item Tester:\\
	M. Schäfer
	\item Datum/Uhrzeit:\\
	24.04.2021 21:20 Uhr
\end{itemize}

\subsection{Testfall 1.2}
\begin{itemize}
	\item Testgegenstand:\\
	Versenden einer E-Mail an einen nicht existierenden Empfänger
	\item Testkonfiguration:\\
	System ohne Last\\
	Mail-Sender \verb+mail@kitanet.local+ in System angelegt\\
	Mail-Empfänger \verb+foobar@kitanet.local+ nicht in System angelegt
	\item Testbeschreibung:\\
	Versenden einer E-Mail über Roundcube mit Nutzer \verb+mail@kitanet.local+ \\ an \verb+foobar@kitanet.local+
	\item Bezug:\\
	Anforderung \textit{(E)SMTP-Kommunikation}
	\item Priorität:\\
	nachgeordnet
	\item Details:\\
	Test provoziert SMTP-Fehlermeldung 550
	\item Soll-Ergebnis:\\
	Sender erhält SMTP-Fehlermeldung 550 oder äquivalente Rückmeldung des Servers \citep[vgl.][16]{rfc821}.
	\item Ist-Ergebnis:\\
	Roundcube meldet Fehler \textit{SMTP-Fehler 550 - 5.1.1. recipient address rejected. User unkown in virtual table}.
	\item Bestanden:\\
	Ja
	\item Aus welcher Phase stammt der Fehler:\\
	Kein Fehler
	\item Kommentar:\\
	Erwartet wurde Fehlermeldung per Mail von SMTP-Server\\
	Roundcube prüft selbstständig Gültigkeit des Empfängers
	\item Tester:\\
	M. Schäfer
	\item Datum/Uhrzeit:\\
	24.04.2021 21:25 Uhr
\end{itemize}

\section{Testreihe 2 LDAP-Anbindung}
\subsection{Testfall 2.1}
\begin{itemize}
	\item Testgegenstand:\\
	Anlegen eines Benutzers via LDAP und Versenden einer Test-Mail
	\item Testkonfiguration:\\
	ruhendes System\\
	Mail-Sender \verb+mail@kitanet.local+ in System angelegt\\
	Mail-Empfänger \verb+ldap-test3@kitanet.local+ nicht in System angelegt
	\item Testbeschreibung:\\
	Nutzer mit LDAP-Attribute \verb+mail = ldap-test3@kitanet.local+ wird im System angelegt\\
	Wartezeit bis Datenaustausch zwischen LDAP und SMTP-Server stattgefunden hat. (ggf. manuelles Anstoßen des Cronjobs)\\
	Versenden einer E-Mail über Dovecot-Client mit Nutzer \verb+mail@kitanet+ \\ an \verb+ldap-test3@kitanet.local+
	\item Bezug:\\
	Anforderung \textit{LDAP-Anbindung}
	\item Priorität:\\
	hoch
	\item Details:\\
	Neue Benutzer sollen E-Mail-Funktionalität ohne Eingreifen des Administrators nutzen können.\\
	Manuelles Auslösen des Cronjobs erspart Wartezeit.
	\item Soll-Ergebnis:\\
	E-Mail geht in Postfach des Empfängers ein.
	\item Ist-Ergebnis:\\
	E-Mail ging in Postfach des Empfängers ein.
	\item Bestanden:\\
	Ja
	\item Aus welcher Phase stammt der Fehler:\\
	Kein Fehler	
	\item Kommentar:\\
	keine Wartezeit, Anstoß Cronjob unnötig. Nutzer sofort erreichbar.
	\item Tester:\\
	M. Schäfer
	\item Datum/Uhrzeit:\\
	24.04.2021 21:30 Uhr
\end{itemize}

\subsection{Testfall 2.2}
\begin{itemize}
	\item Testgegenstand:\\
Löschen eines Benutzers via LDAP und Versenden einer Test-Mail
	\item Testkonfiguration:\\
	Testfall 2.1 wurde erfolgreich durchgeführt\\
	System ruht\\
	Keine Änderung der Konfiguration
	\item Testbeschreibung:\\
	Löschung des Nutzers mit LDAP-Attribut: \verb+mail = ldap-test3@kitanet.local+\\
	Wartezeit bis Datenaustausch zwischen LDAP und SMTP-Server stattgefunden hat. (ggf. manuelles Anstoßen des Cronjobs)\\
	Versenden einer E-Mail über Dovecot-Client mit Nutzer \verb+mail@kitanet.local+ \\ an \verb+ldap-test3@kitanet.local+
	\item Bezug:\\
	Anforderung \textit{LDAP-Anbindung}
	\item Priorität:\\
	hoch
	\item Details:\\
	Löschen alter Nutzer verringert Speicherbedarf auf den Servern.
	\item Soll-Ergebnis:\\
	Sender erhält SMTP-Fehlermeldung 550 oder äquivalente Rückmeldung des Servers \citep[vgl.][16]{rfc821}.
	\item Ist-Ergebnis:\\
	SMTP-Fehler, korrespondierend zu Testfall 1.2
	\item Bestanden:\\
	Ja
	\item Aus welcher Phase stammt der Fehler:\\
	Kein Fehler
	\item Kommentar:\\
	Löschung zeigt sofortige Wirkung. 
	\item Tester:\\
	M. Schäfer
	\item Datum/Uhrzeit:\\
	24.04.2021 21:32 Uhr
\end{itemize}



\section{Systemausfall und Neustart}
\subsection{Testfall 3.1}
\begin{itemize}
	\item Testgegenstand:\\
Kontrollierter Neustart des Systems
	\item Testkonfiguration:\\
	System ruht
	\item Testbeschreibung:\\
	Verbindung zu Server via \verb+SSH+
	Befehl zum Neustart wird erteilt (\verb+sudo reboot+)
	Nachdem Weboberfläche von \textit{KitaNet} wieder erreichbar ist, wird Testfall 1.1 wiederholt.
	\item Bezug:\\
	Anforderung \textit{Zuverlässigkeit}
	\item Priorität:\\
	unbedingt erforderlich
	\item Details:\\
	Aufwand für Wiederinbetriebnahme des Systems soll so gering wie möglich gehalten werden. SMTP-Software soll selbstständig starten.
	\item Soll-Ergebnis:\\
	Wiederholung von Testfall 1.1 wird erfolgreich durchgeführt
	\item Ist-Ergebnis:\\
	Mail geht in Postfach des Empfängers ein.
	\item Bestanden:\\
	Ja
	\item Aus welcher Phase stammt der Fehler:\\
	Kein Fehler
	\item Kommentar:\\
	Server nach 36 Sekunden wieder erreichbar.
	Webseite erst nach ca. 1:30 Minuten erreichbar.
	\item Tester:\\
	M. Schäfer
	\item Datum/Uhrzeit:\\
	24.04.2021 21:40 Uhr
\end{itemize}

\subsection{Testfall 3.2}
\begin{itemize}
	\item Testgegenstand:\\
Neustart des Systems nach unkontrollierter Abschaltung
	\item Testkonfiguration:\\
	System ruht
	\item Testbeschreibung:\\
	Trennung des Systems von Stromzufuhr (Stromstecker raus)\\
	Wiederverbindung mit Stromzufuhr\\
	Manueller Neustart des Systems\\
	Nachdem Weboberfläche von \textit{KitaNet} wieder erreichbar ist, wird Testfall 1.1 wiederholt.
	\item Bezug:\\
	Anforderung \textit{Zuverlässigkeit}
	\item Priorität:\\
	unbedingt erforderlich
	\item Details:\\
	System muss nach Stromausfall ohne Eingriff eines Administrators wieder eigenständig  funktionieren.
	\item Soll-Ergebnis:\\
	Wiederholung von Testfall 1.1 wird erfolgreich durchgeführt
	\item Ist-Ergebnis:\\
	Mail geht in Postfach des Empfängers ein.
	\item Bestanden:\\
	Ja
	\item Aus welcher Phase stammt der Fehler:\\
	Kein Fehler
	\item Kommentar:\\
	Server nach 35 Sekunden wieder erreichbar.
	Webserver nach 92 Sekunden wieder erreichbar.
	\item Tester:\\
	M. Schäfer
	\item Datum/Uhrzeit:\\
	24.04.2021 21:45 Uhr
\end{itemize}


\chapter{Dateien und Datenbankeinträge aus Kitanet}
\begin{lstlisting}[caption={Ergebnis der LDAP-Abfrage für den Nutzer 'administrator'}, label={ldapabfrage}]
Enter LDAP Password:
dn: uniqueIdentifier=administrator,ou=people,dc=kitanet
mailAlias: admin@kitanet.local
mailAlias: kitanet@kitanet.local
mailAlias: postmaster@kitanet.local
cn: admin
objectClass: organizationalPerson
objectClass: person
objectClass: top
objectClass: PostfixBookMailAccount
objectClass: extensibleObject
mailUidNumber: 5000
givenName: admin
mailEnabled: TRUE
mailGidNumber: 5000
sn: kitanet
mailQuota: 10240
uniqueIdentifier: administrator
userPassword:: e0NSWVBUfSQ2JGZQbGNTaHVaJEpXRnpRcklyT0FNZ2pWN3ZkT1hhekhiZ3JhQ3B
 MVUlsUFo0OHFCQVpZQ3FDL2MyY0liVkZISmpFT0p1VVJuRU1jeHpnS2Q0WFplR0hNNE84ZFdwREIv
mail: administrator@kitanet.local
mailHomeDirectory: /srv/vmail/administrator@kitanet.local
mailStorageDirectory: maildir:/srv/vmail/administrator@kitanet.local/Maildir
\end{lstlisting}
\newpage
\begin{lstlisting}[caption={/etc/postfix/main.cf}, label={postfix/main.cf}]
### Base Settings
# Listen on all interfaces
	inet_interfaces = all
# Use TCP IPv4 and IPv6
	inet_protocols = all
# Greet connecting clients with this banner
	smtpd_banner = $myhostname ESMTP $mail_name (Ubuntu)
# Fully-qualified hostname
	myhostname = mail.kitanet.local
# Trusted networks/hosts (these are allowed to relay without authentication)
	mynetworks =
    # Local
    localhost
    127.0.0.0/8
    # External
    192.168.4.0/24


### Local Transport
	# Disable local transport (so that system accounts can not receive mail)
		local_transport = error:NO LOCAL
	# Do not use local maps for alias
		alias_maps = 
	# Local domain (could be omitted, since it is automatically derived from $myhostname)
		mydomain = example.com
	# Mails for these domains will be transported locally
		mydestination =
		  $myhostname
		  localhost.$mydomain
		  localhost
# Virtual Reciepients
	# Deliver mail for virtual recipients via Dovecot
		virtual_transport = dovecot
	# Process one mail at one time
		dovecot_destination_recipient_limit = 1
	# accepted virtual domains
		virtual_mailbox_domains = hash:/etc/postfix/virtual_domains
	# accepted virtual recipients
		virtual_mailbox_maps = proxy:ldap:/etc/postfix/ldap_virtual_recipients.cf
	# accepted aliases
		virtual_alias_maps = proxy:ldap:/etc/postfix/ldap_virtual_aliases.cf


### ESMTP 
# SASL
# Enable SASL (required for SMTP authentication)
	smtpd_sasl_auth_enable = yes

### Session Policies  
# Who could receive mails
	smtpd_recipient_restrictions =
    reject_non_fqdn_recipient
    reject_unknown_recipient_domain
  # Allow relaying for SASL authenticated clients and trusted hosts/networks
  # This can be put to smtpd_relay_restrictions in Postfix 2.10 and later 
  	permit_sasl_authenticated
  	permit_mynetworks
  # If not authenticated or on mynetworks, reject mailing to external addresses
 	 reject_unauth_destination
  # Reject the following hosts
  # maybe never used
    check_sender_ns_access cidr:/etc/postfix/drop.cidr
    check_sender_mx_access cidr:/etc/postfix/drop.cidr
  # Finally permit (relaying still requires SASL auth)
  # WARNING: Due to this permit, everyone will be able to send emails to internal addresses without authentication. If this is set to reject though, the server does not receive emails from external addresses. Unfortunately I do not have a solution for this.
	  permit
# Reject the request if the sender is the null address and there are multiple recipients
	smtpd_data_restrictions = reject_multi_recipient_bounce
# who could send mails
	smtpd_sender_restrictions =
    reject_non_fqdn_sender
    reject_unknown_sender_domain
# HELO/EHLO Restrictions
	smtpd_helo_restrictions =
		# mynetworks can communicate with SMTP
      permit_mynetworks
    # look for identitycheck to prevent Hostname abuse
    # maybe never used, but you never know
      check_helo_access pcre:/etc/postfix/identitycheck.pcre
    #reject_non_fqdn_helo_hostname
      reject_invalid_hostname
# Deny VRFY recipient checks
	disable_vrfy_command = yes
# Require HELO
	smtpd_helo_required = yes
# Reject instantly if a restriction applies (do not wait until RCPT TO)
	smtpd_delay_reject = no
# Reject the following Clients (Blacklist)
	smtpd_client_restrictions = check_client_access cidr:/etc/postfix/drop.cidr
\end{lstlisting}

\begin{lstlisting}[caption={/etc/postfix/ldap\_virtual\_recipients.cf}, label={postfix/recipients.cf}]
	bind = yes
# LDAP-User
	bind_dn = uid=postfix,ou=services,dc=kitanet
# LDAP-User-PW
	bind_pw = postfix
# LDAP-Server
	server_host = ldap://127.0.0.1:389
# part of tree, where to search for users
	search_base = ou=people,dc=kitanet
# fqdn for users
	domain = kitanet.local
# LDAP-Query to find the users
# %s is the variable representing the "RCPT TO:" adress
	query_filter = (&(mail=%s)(mailEnabled=TRUE))
# which attribute should be returned
	result_attribute = mail
\end{lstlisting}

\begin{lstlisting}[caption={/etc/postfix/ldap\_virtual\_aliases.cf}, label={postfix/aliases.cf}]
	bind = yes
# LDAP-User
	bind_dn = uid=postfix,ou=services,dc=kitanet
# LDAP-User-PW
	bind_pw = postfix
# LDAP-Server
	server_host = ldap://127.0.0.1:389
# part of tree, where to search for users
	search_base = ou=people,dc=kitanet
# fqdn for users
	domain = kitanet.local
# LDAP-Query to find the users
# %s is the variable representing the "RCPT TO:" adress
	query_filter = (&(mailAlias=%s)(mailEnabled=TRUE))
# which attribute should be returned
	result_attribute = mail
\end{lstlisting}
\begin{lstlisting}[caption={/etc/postfix/sasl/smtpd.conf}, label={postfix/sasl}]
log_level: 3
pwcheck_method: saslauthd
mech_list: PLAIN LOGIN
\end{lstlisting}